\documentclass[11pt]{article}
    \title{\textbf{Práctica 1}}
    \author{Israel Gómez Urbano}
    \date{}
    
    \addtolength{\topmargin}{-3cm}
    \addtolength{\textheight}{3cm}
\begin{document}

\maketitle
\thispagestyle{empty}

\section{Ejercicio 1}
En este caso, vamos a explicar paso a paso, el conjunto potencia de un conjunto.
Aprovechando el ejemplo dado: $ R^3 of R = {\left\lbrace(1, 1),(1, 2),(2, 3),(3, 4)\right\rbrace} $
\\


$ R^2 = R x R $, tal que podemos deducir fácilmente que: $ R^3 = R^2 x R $
\\
\\
Desarrollando este proceso, nos quedaría tal que así:
\\

$R^2 = {\left\lbrace(1, 1),(1, 2),(2, 3),(3, 4)\right\rbrace} x  {\left\lbrace(1, 1),(1, 2),(2, 3),(3, 4)\right\rbrace} $
\\

$R^2 = {\left\lbrace(1, 1),(1, 2),(1, 3),(2, 4)\right\rbrace} $
\\
\\
Y ahora, una vez obtenido $R^2$, pasamos a completar $R^3$
\\

$R^3 = {\left\lbrace(1, 1),(1, 2),(1, 3),(2, 4)\right\rbrace} x  {\left\lbrace(1, 1),(1, 2),(2, 3),(3, 4)\right\rbrace} $
\\

$R^3 = {\left\lbrace(1, 1),(1, 2),(1, 3),(1, 4)\right\rbrace} $

\section{Ejercicio 2}
A través del uso de los comandos, hemos encontrado el archivo mainP.tex
\\


Consideremos $L=\{w\in \{a,b\}^* : w \textnormal{ no termina en } ab\}$. Un expresión regular que genera L es: $(a+b)a+(a+b)^* bb$


\end{document}

